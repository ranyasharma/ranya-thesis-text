\section{Background}\label{sec:background}

\subsection{The Domain Name System}

The Domain Name System (DNS) is a critical component of the Internet’s infrastructure that translates human-readable domain names (e.g., google.com) into Internet Protocol (IP)
addresses. Most Internet communications begin with a client device sending DNS queries to a recursive resolver, which in turn queries one or more name servers, which ultimately refer the client to a server who can map the domain to an IP address. In practice, this resolution process may involve multiple steps: the recursive resolver may first consult its local cache; if no cached answer exists, it may query root servers, top-level domain (TLD) servers, and finally authoritative name servers for the queried domain before returning a response to the client.

The response times of these queries—the time to contact a recursive resolver, query various name servers, and return the results—is important because the DNS underlies virtually all communication on the Internet. For example, when loading a web page, a browser must first resolve the domain names for each object on the page (e.g., the main HTML document, images, stylesheets, JavaScript files, and third-party content) before the objects themselves can be retrieved and rendered. Modern pages can depend on tens or even hundreds of such objects, so even modest DNS delays can accumulate. Thus, the performance of DNS lookup is of utmost importance to application performance such as web performance, as slow DNS lookup times will lead to slow web page loads and degraded user experience. 

\subsection{Privacy Concerns} 

DNS did not originally take privacy and security into account: DNS queries have historically been unencrypted, leaving users susceptible to eavesdropping~\cite{Schmitt_2019}; queries can also be intercepted and manipulated~\cite{Jones2016DetectingDR}. Any on-path entity—such as an ISP, a Wi-Fi hotspot operator, or a malicious observer—can inspect plaintext DNS queries to learn which sites and services a user is accessing and, in some cases, tamper with responses (e.g., for censorship, redirection, or phishing). Previous research has shown that observing a user’s DNS queries can allow users to be tracked across multiple websites. Because these queries are typically made in plaintext, anyone who can observe a user’s network traffic can see not only that a query occurred but also key fields such as the queried domain name (QNAME) and the source IP address. The QNAME contains the full hostname requested by the user and can reveal detailed information about user behavior and communication relationships. For example, lookups for MX records of a small domain can indicate whom a user is emailing, and lookups for certain “sensitive” domains (e.g., controversial organizations or medical services) can be highly revealing. QNAMEs can also leak information about the software or protocols a user is running (e.g., service-specific SRV records), and proposed future uses of DNS for more sensitive data would further increase the privacy risks if queries remain observable in plaintext. The source IP address (and, in practice, often the source port) links these QNAMEs to a particular host or subscriber, allowing observers to build per-user or per-household profiles over time; even when recursive resolvers sit between clients and authoritative servers, mechanisms such as client-subnet extensions or per-user resolvers can re-expose user-specific addressing information. Together, the combination of hostnames (QNAMEs) and source addresses forms rich metadata that enables powerful tracking and re-identification attacks: prior work has demonstrated that an adversary who can observe only the sequence of hostnames over time can reliably link multiple sessions belonging to the same user, even when IP addresses change, by learning distinctive per-user traffic patterns~\cite{rfc7626, herrmann2012reidentification}.

\subsection{Encrypted DNS}

To mitigate security and privacy issues in DNS, operators and vendors have deployed DoT and DoH—both encrypt DNS queries but differ subtly in implementation~\cite{rfc8484, rfc7858}. Encryption prevents passive eavesdroppers from reading users’ DNS payloads. DoT transmits queries over a TLS connection (commonly on port 853), whereas DoH carries DNS over HTTPS (typically on port 443). Because DoT uses a dedicated port, it is generally easier to identify and monitor (and potentially block); by contrast, DoH often blends with other HTTPS traffic and is therefore harder to distinguish at the network level~\cite{cloudflare-dns-over-tls}. Although both protocols protect the contents of DNS queries and responses, they remain susceptible to inference attacks that exploit side channels—e.g., packet sizes, timing and burst patterns, and connection metadata—and, in some cases, to downgrade-style manipulations. In this work, I focus specifically on traffic-analysis inference: what an observer can still learn even when DNS payloads are encrypted.


Most recently, the DNS-over-QUIC, defined in RFC 9250, was rolled out~\cite{rfc9250}. QUIC protocol stands for “Quick UDP internet connections,” and is a transport mechanism standardized by the IETF in 2021. It not only strengthens security, but it also improves speed by combining connection and encryption processes simultaneously. With DoT and DoH, the process takes additional steps compared to DNS over QUIC since DoT has to first establish the connection between the client and the server and then use TLS to encrypt the traffic. Meanwhile, DoQ protocol combines the connection setup and encryption steps, reducing the time it takes to connect to the server. The user initiates a request and the browser sends a DNS query to the DoQ resolver. A QUIC connection is established during this step, ensuring the query is encrypted. The resolver looks into its cache to see if it already has the IP address for the requested domain name. If found, it immediately sends the cached IP address back to the browser. If not cached, the resolver queries the DNS system. The authoritative nameserver resolves the query and the resolver sends the IP address back to the browser. The DoQ resolver receives the IP address and sends it back to the device’s DNS client over the encrypted QUIC connection. If the browser needs to resolve multiple domain names, it can send multiple queries over the same QUIC connection without waiting for previous DNS responses. QUIC’s correction error system retransmits any lost data packets without requiring a full restart of the connection.

In October 2022, the Association for Computing Machinery (ACM) reported a performance study that ran distributed DoQ experiments from multiple vantage points to assess its effect on web browsing~\cite{10.1145/3517745.3561445}. The authors found that DoQ reduced page load times by about 10\% relative to DoH, and—despite the extra encryption—was only about 2\% slower than classic UDP-based DNS.


\subsection{Browser Fingerprinting}

At the application layer, browser fingerprinting aggregates traits—HTTP headers, language and font hints, screen geometry, and protocol signatures such as TLS ClientHello ordering, cipher/extension sets, HTTP/2 settings, or QUIC transport parameters. Even in the absence of cookies, these cross-layer signals can narrow a client’s anonymity set or uniquely identify it across sessions and sites. At the network layer, fingerprinting infers who is communicating and what they are doing from transport- and flow-level metadata: packet size distributions, inter-arrival times (IATs) and burst structure, flow duration, concurrency, retransmissions, and resolver/upstream infrastructure choices (IPs, ASNs, CDNs). In the context of encrypted DNS (DoT/DoH/DoQ), payloads are opaque, but timing and size patterns remain~\cite{laperdrix2019browserfingerprintingsurvey}. 
