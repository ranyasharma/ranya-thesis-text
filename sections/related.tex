\section{Related Work}\label{sec:related}

Li et al. provide a structural view of the encrypted DNS ecosystem, asking not what leaks on the wire, 
but who encrypted resolvers ultimately dpepend on, specifically third parties~\cite{10.1007/978-3-031-70890-9_3}. Through a hierarchical classificaiton 
of DoT/DoH/DoQ servers and large-scale Internet measurements, they separate front-end encrypted 
entrances from the upstream resolvers that answer queries. They introduce a Service Reliance Index (SRI) to quantify 
how strongly a given encrypted resolver depends on specfic upstream DNS providers and find that despite many distinct encrypted
endpoints, rougly 75\% of resolvers forward to simply the top 10 providers. This indicates substantial centralization and dependence on a
small subset of third party DNS services. Complementing this work, Sharma and Feamster measure the availability and response times of a large set of public DoH resolvers
from global vantage points in North America, Europe, and Asia. They find that mainstream providers both perform better and are more
frequently exposed in client configurations, while many non-mainstream resolvers exhibit higher latency or limited replication,
effectively steering users toward a small set of large operators even when more resolvers exist in principle. Taken together with Li et al.,
this work suggests that encrypted DNS is centralized both in terms of infrastructure and effective performance, heightening the impact of any
protocol or traffic-level leakage that this work seeks to characterize~\cite{sharma2025measuringavailabilityresponsetimes}. 

Garcia et al. conduct a large scale, longitudinal measurement of DoH, DoT, and DoQ adoption using trafffic from an ISP 
backbone, a large university network, and a global security company, finding that in early 2021, encrypted DNS accounted for only 0.01\% of total 
DNS traffic~\cite{garcía2021largescalemeasurementadoption}. Usage levels were statistically stationary during that five-month window.
Their study is the first to provide a comprehensive, cross-protocol analysis of DoH, DoT, and DoQ trends, combining traffic measurements with active
Internet scans to identify previously unpublished DoH resolvers. However, their analysis focuses on aggregate traffic volumes and protocol 
mix rather than user or site identifiable information. Aditionally, the measurements predate more recent deployment changes in browsers and operating systems.

Hu and Fukuda study privacy leakage in DoQ by mounting website fingerprinting attacks against encrypted DNS traffic under two realistic configurations:
a DNS proxy (BIND) forwarding to a local resolver, and direct use of the public encrypted resolver NextDNS~\cite{10463369}. They show that an on-path adversary can still classify 
whether a visited site belongs to a sensitive category with high accuracy, and they identify packet interarrival times and packed lengths as the most discriminative features. 
Building on this, they evaluate countermeasures such as adding random response delays and padding DNS payloads, finding that these defenses can reduce mean F1 scores
by 20-25\%, at the cost of additional latency. This work is methodologically close to ours; however, their study is restricted to DoQ, whereas our work provides a cross protocl 
comparison (DoT/DoH/DoQ). 

Zhan et al. take a broader QUIC-centric view and analyze website fingerprinting attacks on early GQUIC, IQUIC, and HTTPS, showing that in "early traffic" scenarios, an adversary can reach 
over 95\% accuracy on QUIC with only ~40 packets using simple traffic features, and that QUIC can be more vulnerable than HTTPS~\cite{ZHAN2021108538}. THey demonstrate that many 
fingerpinting features transfer across transport protocols in full-traffic scenarios, suggesting that protocol changes alone do not eliminate fingerprintability.
While their work targets end-to-end web flows rather than DNS specifically, it reinforces a key theme of this thesis: even when payloads are encrypted and protocols upgraded,
early-flow metadata and timing patterns remain highly informative, motivating a systematic accounting of which DNS features are exposed under different encrypted DNS designs.

